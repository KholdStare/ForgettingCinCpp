\documentclass[table]{beamer}

%\usetheme[secheader]{Boadilla}
\usetheme{Boadilla}
\usecolortheme{beetle}
\usepackage[latin1]{inputenc}
\usepackage{listings} % for code syntax highlighting
\usepackage{caption}

\usepackage{color}
\usepackage{booktabs} % better tables
\usepackage{colortbl} % for coloring tables

%%%%%%%%%%%%%%%%%%%%%%%%%%%%%%%%%%%%%%%%%%%%%%%%%%%%%%%%%%%%%%%%%%%%%%%%%%%
%{{{                               Fonts                                  %
%%%%%%%%%%%%%%%%%%%%%%%%%%%%%%%%%%%%%%%%%%%%%%%%%%%%%%%%%%%%%%%%%%%%%%%%%%%


\usepackage{inconsolata} % for code
\usepackage{lmodern} % normal text
\renewcommand*\familydefault{\sfdefault} 
\usepackage[T1]{fontenc}
%}}}

%%%%%%%%%%%%%%%%%%%%%%%%%%%%%%%%%%%%%%%%%%%%%%%%%%%%%%%%%%%%%%%%%%%%%%%%%%%
%{{{                      Beamer Theme and colors                         %
%%%%%%%%%%%%%%%%%%%%%%%%%%%%%%%%%%%%%%%%%%%%%%%%%%%%%%%%%%%%%%%%%%%%%%%%%%%

 
\definecolor{themegray}{HTML}{999999}
\definecolor{codebg}{HTML}{2C2C2C} % dark gray
\definecolor{codefg}{HTML}{D0D0D0} % light gray
\definecolor{themeblue}{HTML}{789FEF} % light blue
\definecolor{codecomment}{HTML}{808080} % gray
\definecolor{codestring}{HTML}{DFDF87} % for strings, light yellow
\definecolor{themedarkblue}{HTML}{1D2561}
\definecolor{themegreen}{HTML}{6ABF29}
\definecolor{themeyellow}{HTML}{DFDF67}
\definecolor{themered}{HTML}{E71717}
\definecolor{themeorange}{HTML}{DF8F27}

% tweak beamer theme
\setbeamercolor{framesubtitle}{fg=white}
\setbeamercolor{frametitle}{bg=codebg}
\setbeamercolor{title}{bg=codebg}
\setbeamercolor{palette primary}{bg=codebg}
\setbeamercolor{author}{fg=white}
\setbeamercolor{date}{fg=white}
\setbeamercolor{alerted text}{fg=themered}
\setbeamercolor{palette primary}{fg=codefg}
%}}}

%%%%%%%%%%%%%%%%%%%%%%%%%%%%%%%%%%%%%%%%%%%%%%%%%%%%%%%%%%%%%%%%%%%%%%%%%%%
%{{{                           Tweak Tables                               %
%%%%%%%%%%%%%%%%%%%%%%%%%%%%%%%%%%%%%%%%%%%%%%%%%%%%%%%%%%%%%%%%%%%%%%%%%%%

\setlength{\aboverulesep}{0pt}
\setlength{\belowrulesep}{0pt}
\setlength{\extrarowheight}{.75ex}
%}}}
 
%%%%%%%%%%%%%%%%%%%%%%%%%%%%%%%%%%%%%%%%%%%%%%%%%%%%%%%%%%%%%%%%%%%%%%%%%%%
%{{{                          Tweak Listings                              %
%%%%%%%%%%%%%%%%%%%%%%%%%%%%%%%%%%%%%%%%%%%%%%%%%%%%%%%%%%%%%%%%%%%%%%%%%%%

% caption stuff
\DeclareCaptionFormat{listing}{\parbox{\textwidth}{#1#2#3}}
\captionsetup[lstlisting]{format=listing}

\lstset{ %
  language=C++,                % the language of the code
  basicstyle=\small\ttfamily\color{codefg},           % the size of the fonts that are used for the code
  backgroundcolor=\color{codebg},      % choose the background color. You must add \usepackage{color}
  showspaces=false,               % show spaces adding particular underscores
  showstringspaces=false,         % underline spaces within strings
  showtabs=false,                 % show tabs within strings adding particular underscores
  %frame=single,                   % adds a frame around the code
  rulecolor=\color{black},        % if not set, the frame-color may be changed on line-breaks within not-black text (e.g. commens (green here))
  tabsize=2,                      % sets default tabsize to 2 spaces
  captionpos=t,                   % sets the caption-position to bottom
  breaklines=true,                % sets automatic line breaking
  breakatwhitespace=false,        % sets if automatic breaks should only happen at whitespace
  title=\lstname,                   % show the filename of files included with \lstinputlisting;
                                  % also try caption instead of title
  keywordstyle=\color{themeblue}\bfseries,
  stringstyle=\color{codestring},
  commentstyle=\color{codecomment},
  escapeinside={\%*}{*)},            % if you want to add LaTeX within your code
  morekeywords={*,...}               % if you want to add more keywords to the set
}
% ========================================== }}}

%%%%%%%%%%%%%%%%%%%%%%%%%%%%%%%%%%%%%%%%%%%%%%%%%%%%%%%%%%%%%%%%%%%%%%%%%%%
%{{{                  Custom Macros for Presentation                      %
%%%%%%%%%%%%%%%%%%%%%%%%%%%%%%%%%%%%%%%%%%%%%%%%%%%%%%%%%%%%%%%%%%%%%%%%%%%

% define a counter for rules
\newcounter{rulecount}
\newcommand{\declarerule}{\textbf{\color{themeblue}{Rule \therulecount:}} }

\newcommand{\declarelesson}{\textbf{\color{themegreen}{Lesson:}} }

% behind the scenes title
\newcommand{\declarebts}{\textbf{\color{themeorange}{Behind the Scenes:}} }
%}}}

%%%%%%%%%%%%%%%%%%%%%%%%%%%%%%%%%%%%%%%%%%%%%%%%%%%%%%%%%%%%%%%%%%%%%%%%%%%
%{{{                            Author info                               %
%%%%%%%%%%%%%%%%%%%%%%%%%%%%%%%%%%%%%%%%%%%%%%%%%%%%%%%%%%%%%%%%%%%%%%%%%%%


\author{Alexander Kondratskiy}
\date{\today}
% \institute[2008]{ECON 101}

%}}}
%%%%%%%%%%%%%%%%%%%%%%%%%%%%%%%%%%%%%%%%%%%%%%%%%%%%%%%%%%%%%%%%%%%%%%%%%%%

% Modeline for vim settings:
% vim60:fdm=marker:


\setbeamercolor{frametitle}{fg=themeblue}
\setbeamercolor{title}{fg=themeblue}

\title{Automatic Resource Management in C++}


%%%%%%%%%%%%%%%%%%%%%%%%%%%%%%%%%%%%%%%%%%%%%%%%%%%%%%%%%%%%%%%%%%%%%%%%%%%
%{{{                         Main Presentation                            %
%%%%%%%%%%%%%%%%%%%%%%%%%%%%%%%%%%%%%%%%%%%%%%%%%%%%%%%%%%%%%%%%%%%%%%%%%%%

\begin{document}

\frame{\titlepage}

\section{Introduction}
\frame{\sectionpage}

% trick question
\begin{frame}[fragile]
    \frametitle{Manual Resource Management}
    \begin{lstlisting}[title=See anything wrong?]
void process(const char* filename) {
    FILE* fp = fopen(filename, "r");




    do_file_stuff(fp);





    fclose(fp);
}
    \end{lstlisting}
\end{frame}

% trick question 2
\begin{frame}[fragile]
    \frametitle{Manual Resource Management}
    \begin{lstlisting}[title=See anything wrong?]
void process(const char* filename) {
    FILE* fp = fopen(filename, "r");



    // lots of code ...


    // more code ...



    fclose(fp);
}
    \end{lstlisting}
\end{frame}

% file leak
\begin{frame}[fragile]
    \frametitle{Manual Resource Management}
    \begin{lstlisting}[title=Forgetting to close!]
void process(const char* filename) {
    FILE* fp = fopen(filename, "r");

    // other code ...

    // error handling
    if (someCondition) {
        return;
    }

    // more code ...

    fclose(fp);
}
    \end{lstlisting}
\end{frame}

% manual should set off alarm
\begin{frame}[fragile]
    \frametitle{Manual Resource Management}
    \begin{itemize}
        \item Explicit management should set off an alarm
            \begin{itemize}
                \item no more \texttt{fclose}
                \item no more \texttt{delete}
            \end{itemize}
        \item I hope to convince you of a better way
    \end{itemize}
\end{frame}

% error prone
\begin{frame}
    \frametitle{Manual Resource Management}
    \framesubtitle{Error prone}
    \begin{itemize}
        \item Many examples of resources
            \begin{itemize}
                \item Memory allocation
                \item IO handles
                \item Mutexes
                \item Socket connections
                \item Domain specific resources
            \end{itemize}
        \item Many ways to fail
            \begin{itemize}
                \item Early exit via \texttt{return} or \texttt{break}
                \item Exceptions
                \item No explicit ownership (who deallocates?)
            \end{itemize}
        \item Manual resource management ala C is very error prone!
            \begin{itemize}
                \item Doubly so with exception in C++!
            \end{itemize}
    \end{itemize}
\end{frame}


%}}}

% TODO: composability

% TODO: RAII epic sauce
%%%%%%%%%%%%%%%%%%%%%%%%%%%%%%%%%%%%%%%%%%%%%%%%%%%%%%%%%%%%%%%%%%%%%%%%%%%
%{{{                               RAII                                   %
%%%%%%%%%%%%%%%%%%%%%%%%%%%%%%%%%%%%%%%%%%%%%%%%%%%%%%%%%%%%%%%%%%%%%%%%%%%

\subsection{RAII resource management}
\frame{\subsectionpage}

\begin{frame}
    \frametitle{RAII}
    \begin{itemize}
        \item<1->Thankfully a resource is already managed for us
            \begin{itemize}
                \item<2->The stack!
            \end{itemize}
        \item<2->Destructor called at end of scope
        \item<2->Can we exploit this?
    \end{itemize}
\end{frame}

% towards RAII
\begin{frame}
    \frametitle{RAII}
    \begin{itemize}
        \item In previous situation, need to ensure file closes when exiting function
        \item What is triggered on scope exit?
            \begin{itemize}
                \item Local values are destroyed
            \end{itemize}
        \item Exploit destructor to close the file for us!
        \item That is RAII
            \begin{itemize}
                \item Resource Acquisition Is Initialization
                \item Bjarne admits it's a bad name :P
            \end{itemize}
        \item Let's do it!
    \end{itemize}
\end{frame}

% file handle class
\begin{frame}[fragile]
    \frametitle{RAII}
    \framesubtitle{RAII class to handle files}
    \begin{lstlisting}[title=Problems begone!]
class FileHandle {
    FILE* fp;

public:
    FileHandle(const char* filename) :
        fp( fopen(filename, "r") ) { }

    ~FileHandle() {
        fclose(fp);
    }

    FILE* get() const {
        return fp;
    }
}
    \end{lstlisting}
\end{frame}

% using RAII object
\begin{frame}[fragile]
    \frametitle{RAII}
    \framesubtitle{RAII class to handle files}
    \begin{lstlisting}[title=Problems begone!]
void process(const char* filename) {
    FileHandle(filename);

    // as many returns and exceptions
    // as your heart desires
}
    \end{lstlisting}
\end{frame}

% quick summary
\begin{frame}
    \frametitle{RAII}
    \framesubtitle{Summary}
    \begin{itemize}
        \item<1->Resource Acquisition Is Initialization 
        \item<1->Exploit object lifetime and destructors
        \item<1->Put all resource cleanup in destructors
        \item<1->Let the compiler clean up for you
        \item<2->But what about memory?
    \end{itemize}
\end{frame}
%}}}

\end{document}

% Modeline for vim settings:
% vim60:fdm=marker:
